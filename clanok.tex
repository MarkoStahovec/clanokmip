\documentclass[10pt,twoside,slovak,a4paper]{article}

\usepackage[slovak]{babel}
%\usepackage[T1]{fontenc}
\usepackage[IL2]{fontenc} % lepšia sadzba písmena Ľ než v T1
\usepackage[utf8]{inputenc}
\usepackage{graphicx}
\usepackage{url} % príkaz \url na formátovanie URL
\usepackage{hyperref} % odkazy v texte budú aktívne (pri niektorých triedach dokumentov spôsobuje posun textu)

\usepackage{cite}
%\usepackage{times}

\pagestyle{headings}

\title{Učenie sa cudzích jazykov prostredníctvom mobilu\thanks{Semestrálny projekt v predmete Metódy inžinierskej práce, ak. rok 2020/21, vedenie: Fedor Lehocki}} % meno a priezvisko vyučujúceho na cvičeniach

\author{Marko Stahovec\\[2pt]
	{\small Slovenská technická univerzita v Bratislave}\\
	{\small Fakulta informatiky a informačných technológií}\\
	{\small \texttt{xstahovec@stuba.sk}}
	}

\date{\small 15. október 2020}



\begin{document}

\maketitle

\begin{abstract}
Mobilné zariadenia sa postupom času stali neoddeliteľnou súčasťou našich životov a našej spoločnosti ako celku. Od zariadenia, ktoré slúžilo výlučne na telefonovanie a prenos krátkych textových správ, sa vyvinula pomôcka, vďaka ktorej je náš život mnohonásobne jednoduchší a vďaka ktorej sa dá veľa naučiť. Hlavná charakteristika mobile-learningu je ponímaná ako spôsob výučby, ktorý je spontánny, neformálny a všadeprítomný. Síce takýto spôsob výučby nemusí byť práve najefektívnejší, no učiacemu ponúka široké spektrum možností vrátane slobody, času a predovšetkým priestoru, keďže mobilný telefón je použiteľný takmer v akejkoľvek situácii. A preto sa v tejto práci budeme venovať kladom a záporom, t.j. ľahkej dostupnosti, individuálnosti, ale aj nevýhodám malej obrazovky, ukladaniu dát apod. Taktiež sa dotkneme spôsobov, akými by sa dalo také učenie realizovať a taktiež schopnostiam, ktoré sa daným štýlom učenia môžu bohato rozvíjať, napr. gramatika, porozumenie, slovná zásoba apod.
\end{abstract}



\section{Úvod}

Motivujte čitateľa a vysvetlite, o čom píšete. Úvod sa väčšinou nedelí na časti.

Uveďte explicitne štruktúru článku. Tu je nejaký príklad.
Základný problém, ktorý bol naznačený v úvode, je podrobnejšie vysvetlený v časti~\ref{nejaka}.
Dôležité súvislosti sú uvedené v častiach~\ref{dolezita} a~\ref{dolezitejsia}.
Záverečné poznámky prináša časť~\ref{zaver}.



\section{Nejaká časť} \label{nejaka}

Z obr.~\ref{f:rozhod} je všetko jasné. 

\begin{figure*}[tbh]
\centering
%\includegraphics[scale=1.0]{diagram.pdf}
Aj text môže byť prezentovaný ako obrázok. Stane sa z neho označný plávajúci objekt. Po vytvorení diagramu zrušte znak \texttt{\%} pred príkazom \verb|\includegraphics| označte tento riadok ako komentár (tiež pomocou znaku \texttt{\%}).
\caption{Rozhodujúci argument.}
\label{f:rozhod}
\end{figure*}



\section{Iná časť} \label{ina}

Základným problémom je teda\ldots{} Najprv sa pozrieme na nejaké vysvetlenie (časť~\ref{ina:nejake}), a potom na ešte nejaké (časť~\ref{ina:nejake}).\footnote{Niekedy môžete potrebovať aj poznámku pod čiarou.}

Môže sa zdať, že problém vlastne nejestvuje\cite{Miangah2012}, ale bolo dokázané, že to tak nie je~\cite{KukulskaHulme2009}. Napriek tomu, aj dnes na webe narazíme na všelijaké pochybné názory\cite{Kim2012}. Dôležité veci možno \emph{zdôrazniť kurzívou}.


\subsection{Nejaké vysvetlenie} \label{ina:nejake}

Niekedy treba uviesť zoznam:

\begin{enumerate}
\item jedna vec
\item druhá vec
	\begin{enumerate}
	\item x
	\item y
	\end{enumerate}
\end{enumerate}


\subsection{Ešte nejaké vysvetlenie} \label{ina:este}

\paragraph{Veľmi dôležitá poznámka.}
Niekedy je potrebné nadpisom označiť odsek. Text pokračuje hneď za nadpisom.



\section{Dôležitá časť} \label{dolezita}

Je potrebný aj obrázok/graf/tabuľka.



\section{Ešte dôležitejšia časť} \label{dolezitejsia}




\section{Záver} \label{zaver} % prípadne iný variant názvu



%\acknowledgement{Ak niekomu chcete poďakovať}


% týmto sa generuje zoznam literatúry z obsahu súboru literatura.bib podľa toho, na čo sa v článku odkazujete
\bibliography{literatura}
\bibliographystyle{plain} % prípadne alpha, abbrv alebo hociktorý iný
\end{document}
